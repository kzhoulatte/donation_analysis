%
% This is a borrowed LaTeX template file for lecture notes for CS267,
% Applications of Parallel Computing, UCBerkeley EECS Department.
% Now being used for CMU's 10725 Fall 2012 Optimization course
% taught by Geoff Gordon and Ryan Tibshirani.  When preparing 
% LaTeX notes for this class, please use this template.
%
% To familiarize yourself with this template, the body contains
% some examples of its use.  Look them over.  Then you can
% run LaTeX on this file.  After you have LaTeXed this file then
% you can look over the result either by printing it out with
% dvips or using xdvi. "pdflatex template.tex" should also work.
%

\documentclass[twoside]{article}
\setlength{\oddsidemargin}{0.25 in}
\setlength{\evensidemargin}{-0.25 in}
\setlength{\topmargin}{-0.6 in}
\setlength{\textwidth}{6.5 in}
\setlength{\textheight}{8.5 in}
\setlength{\headsep}{0.75 in}
\setlength{\parindent}{0 in}
\setlength{\parskip}{0.1 in}
\usepackage{graphicx}
\graphicspath{ {images/} }
\usepackage{float}

%
% ADD PACKAGES here:
%

\usepackage{amsmath,amsfonts,graphicx}

%
% The following commands set up the lecnum (lecture number)
% counter and make various numbering schemes work relative
% to the lecture number.
%
\newcounter{lecnum}
\renewcommand{\thepage}{\thelecnum-\arabic{page}}
\renewcommand{\thesection}{\thelecnum.\arabic{section}}
\renewcommand{\theequation}{\thelecnum.\arabic{equation}}
\renewcommand{\thefigure}{\thelecnum.\arabic{figure}}
\renewcommand{\thetable}{\thelecnum.\arabic{table}}

%
% The following macro is used to generate the header.
%
\newcommand{\lecture}[4]{
   \pagestyle{myheadings}
   \thispagestyle{plain}
   \newpage
   \setcounter{lecnum}{#1}
   \setcounter{page}{1}
   \noindent
   \begin{center}
   \framebox{
      \vbox{\vspace{2mm}
    \hbox to 6.28in { {\bf Project: Code Challenge by Insight Data Engineer
	\hfill Winter 2018} }
       \vspace{4mm}
       \hbox to 6.28in { {\Large \hfill Report: Repeat Donors for Multiply Years  \hfill} }
       \vspace{2mm}
       \hbox to 6.28in { {\it Kuan Zhou \hfill Date: Feb 09} }
      \vspace{2mm}}
   }
   \end{center}
   \markboth{}{}

   %{\bf Note}: {\it LaTeX template courtesy of UC Berkeley EECS dept.}

   %{\bf Disclaimer}: {\it These notes have not been subjected to the
   %usual scrutiny reserved for formal publications.  They may be distributed
   %outside this class only with the permission of the Instructor.}
   %\vspace*{4mm}
}
%
% Convention for citations is authors' initials followed by the year.
% For example, to cite a paper by Leighton and Maggs you would type
% \cite{LM89}, and to cite a paper by Strassen you would type \cite{S69}.
% (To avoid bibliography problems, for now we redefine the \cite command.)
% Also commands that create a suitable format for the reference list.
\renewcommand{\cite}[1]{[#1]}
\def\beginrefs{\begin{list}%
        {[\arabic{equation}]}{\usecounter{equation}
         \setlength{\leftmargin}{2.0truecm}\setlength{\labelsep}{0.4truecm}%
         \setlength{\labelwidth}{1.6truecm}}}
\def\endrefs{\end{list}}
\def\bibentry#1{\item[\hbox{[#1]}]}

%Use this command for a figure; it puts a figure in wherever you want it.
%usage: \fig{NUMBER}{SPACE-IN-INCHES}{CAPTION}
\newcommand{\fig}[3]{
			\vspace{#2}
			\begin{center}
			Figure \thelecnum.#1:~#3
			\end{center}
	}
% Use these for theorems, lemmas, proofs, etc.
\newtheorem{theorem}{Theorem}[lecnum]
\newtheorem{lemma}[theorem]{Lemma}
\newtheorem{proposition}[theorem]{Proposition}
\newtheorem{claim}[theorem]{Claim}
\newtheorem{corollary}[theorem]{Corollary}
\newtheorem{definition}[theorem]{Definition}
\newenvironment{proof}{{\bf Proof:}}{\hfill\rule{2mm}{2mm}}

% **** IF YOU WANT TO DEFINE ADDITIONAL MACROS FOR YOURSELF, PUT THEM HERE:

\newcommand\E{\mathbb{E}}

\begin{document}
%FILL IN THE RIGHT INFO.
%\lecture{**LECTURE-NUMBER**}{**DATE**}{**LECTURER**}{**SCRIBE**}
\lecture{1}{Feb 09}{Kuan Zhou}{}
%\footnotetext{These notes are partially based on those of Nigel Mansell.}

% **** YOUR NOTES GO HERE:

% Some general latex examples and examples making use of the
% macros follow.  
%**** IN GENERAL, BE BRIEF. LONG SCRIBE NOTES, NO MATTER HOW WELL WRITTEN,
%**** ARE NEVER READ BY ANYBODY.
%This lecture's notes illustrate some uses of
%various \LaTeX\ macros.  
%Take a look at this and imitate.



This file serves as a report for code challenge - repeat donors for multiple years - designed by Insight Data Engineer, which covers methods, dependencies, conclusions etc. 

\section{ Task Description } % Don't be this informal in your notes!

In order to gain more insights on campaign contributions for political candidates, a file listing individual campaign contributions for multiple years is used to determine which ones came from repeat donors and calculate a few statistics.

\section{ Analysis }

(1) \textbf{Grouped Zip order: }

The major features of this implementation are: 
 \begin{itemize}
\item Use \verb|defaultdict()| with list for DONARS to complete search in O(1) time;
\item Use \verb|defaultdict()| with set for ZIPS to complete search in O(1) time;
\item Use a framework of \verb|zip-donar-donor| to extract the repeat donors and output in a grouped zip code order;
\item Use of \verb|numpy.percentile()| with \verb|interpolation='lower'| for percentile calculation;
\item Donation with amount less than 1 dollars is neglected. 
 \end{itemize}
 
 As in \verb|donation_analytics.py|, the running time can be approximated to be at most $O(n lg(n))$. 
 
 \bigskip
 
 \textbf{RESULT:} For \verb|itcont.txt| of \verb|indiv16| with $3.83G$, running time: $419.71s$
 
  \bigskip
 
 (2) \textbf{Streaming Input order: }
 
 Another method with no grouped zip order is implemented in \verb|donation_analytics_stream.py|. For this implementation, the results for contribution amounts and percentiles are also grouped for zip code and year, but they are calculated and written into output file with streaming input order. 
 
Because the major restriction in my PC is speed not memory, the input is read in and stored in a defaultdict \verb|hash_donars|. When memory is a restriction, it can easily be changed to work in a streaming way. 
 
 As in \verb|donation_analytics_stream.py|, the running time can be approximated to be within $O(n)$. 
 
  \bigskip
 
 \textbf{RESULT:} For \verb|itcont.txt| of \verb|indiv16| with $3.83G$, running time: $448.16s$
 
  \bigskip
 
 \section{Dependencies}

\hspace{10ex} Operating System: Ubuntu 16.04 LTS(Linux) 64 bit

\hspace{10ex} Memory: 31.3 GiB

\hspace{10ex} Processor: Intel Core i7

\hspace{10ex} Packages: Python 3, Numpy. 

\section{ How to Run } % Don't be this informal in your notes!

\bigskip

\hspace{4ex} In \verb|./| directory:

\begin{verbatim}
         ./run.sh
\end{verbatim}

\hspace{4ex} In \verb|./src/| directory:

\begin{verbatim}
         python3 ./donation_analytics.py ../input/itcont.txt ../input/percentile.txt 
           ../output/repeat_donors.txt
\end{verbatim}

\begin{itemize}
\item \verb|./src/donation_analytics.py|: the codes to process the extraction in grouped zip order;
\item \verb|./src/donation_analytics_stream.py|: the codes to process the extraction in streaming order;
\item \verb|./input/|: input directory with \verb|itcont.txt| and \verb|percentile.txt| files;
\item \verb|./output/|: output directory with \verb|repeat_donors.txt| output file.
\end{itemize}
 
  
 \section{Conclusion}
 
 A well organized framework to extract the repeat donors with percentile statistics is completed within $O(n)$ time. 

  \section{Future Improvements}
 \begin{itemize}
 \item \textbf{CPython} Modules in CPython can be used to boost the speed;
 
 \item \textbf{Multi Threads} Multi threads \verb|multiprocessing| module with insights into codes can be used to parallel the computing; 
 
 \item \textbf{PyPy} A new package PyPy with even more speed can be used to further boost the calculation.
 \end{itemize}
 
  \section{Possible Errors}
 \begin{itemize}
 \item Since relation of zips and DONARS are stored using a \verb|defaultdict()| with set, the order of zips and DONARS are in random. The output order can be random for DONARS for each zip. In this case, the results are correct, but not exactly as in test sample. 
 \item Some other malformed input not listed as in task requirements. 
 \end{itemize}

\end{document}





